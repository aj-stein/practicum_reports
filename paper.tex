\documentclass{jdf}

\begin{document}
Section: PUBP-6727
\title{Mutual Monitoring in the Cloud}
\author{Alexander Stein}

\maketitle
\thispagestyle{fancy}

\section{Problem Statement}

Cloud computing infrastructure is essentially ubiquitous, but adoption is not without challenges. Cloud service providers must cater to customers in regulated sectors, complying with cybersecurity frameworks that create high barriers to entry. One barrier is ongoing evaluation of the provider's cybersecurity posture, often resulting in centralized bureaucracies. FedRAMP oversees and documents a prominent example of such a program, the Continuous Monitoring Program.

Are these bureaucracies an optimal solution, or a last resort that fails to keep pace with cloud technology as it proliferates and evolves? If they are a last resort, is there a better way?

\section{Solution Statement}

\section{Completed Tasks (Last 2 Weeks)}

\section{Tasks for the Next Project Report}

\section{Questions or issues I'm having}

\section{Methodology Paragraph Summary}

\section{Timeline}

\begin{tabular}{||l l l||} 
    \hline
    Week \# & Description of Task & Status \\ [0.5ex] 
    \hline
    W1 & Initialize code repository for prototype service & Complete \\
    \hline
    W1 & Present proposal to advisors and integrate feedback; secure advisorship & In Progress \\
    \hline
    W1 & Begin outline of FedRAMP ConMon critical analysis & In Progress \\
    \hline
\end{tabular}

\section{Evaluation}

\section{Report Outline}

\section{References}

\nocite{*}
\bibliographystyle{apacite}
\bibliography{references.bib}

\section{Appendix}

\end{document}
