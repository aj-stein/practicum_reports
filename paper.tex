\documentclass{jdf}

\begin{document}
Section: PUBP-6727
\title{Mutual Monitoring in the Cloud \\ Progress Report 2}
\author{Alexander Stein}

\maketitle
\thispagestyle{fancy}


\section*{Problem Statement}

Cloud computing infrastructure is essentially ubiquitous, but adoption is not without challenges. Cloud service providers must cater to customers in regulated sectors, complying with cybersecurity frameworks that create high barriers to entry. One barrier is ongoing evaluation of the provider's cybersecurity posture, often resulting in centralized bureaucracies. FedRAMP oversees a prominent example of such a program, the Continuous Monitoring Program, which is emblematic of these barriers. This program requires hundreds of cloud service providers to contract with one of thirty reputable auditor firms. The providers work with the auditors to send security scans and updated security control documentation for FedRAMP-authorized services monthly to FedRAMP reviewers, in some cases for the largest cloud infrastructures in the world. All three parties collaborate in meetings, emails, and a wiki, forming a unique multi-party bureaucracy that both secures and bottlenecks the government's acquisition of modern cloud services.

Are these bureaucracies an optimal solution, or a last resort that fails to keep pace with cloud technology as it proliferates and evolves? If they are a last resort, is there a better way?

\section*{Solution Statement}

I will use this research to design and evaluate an alternative to centralized continuous monitoring, mutual monitoring. The foundation of mutual monitoring will be federated data services, known in other security use cases as \hyperlink{https://transparency.dev}{transparency services}. The positives and negatives of FedRAMP's continuous monitoring model will inform its design. Operating such services can change the economics, and thereby the behavior, of cloud service providers and their customers. A new architecture will incentivize auditors to sell value-add analytics via these federated data services, potentially obsoleting centralized authorities for continuous monitoring like FedRAMP.

\section*{Completed Tasks (Last 2 Weeks)}

\begin{enumerate}
    \item I developed a build workflow and automated publication pipeline for the \hyperlink{https://add-architecture-draft--conmotion.netlify.app/architecture.html}{draft architecture specification}. Fixing bugs and troubleshooting took more time than I estimated in my plan.
        \begin{itemize}
            \item I evaluated \hyperlink{https://github.com/martinthomson/i-d-template}{the use of software for authoring specifications popular with IETF standards authors}.
            \item Due to complexity of modifying the aforementioned tooling to remove IETF branding, copyright notices, styling, and features not suitable to my specification, I designed a workflow to author and manage drafts of the specification alongside \hyperlink{https://github.com/aj-stein/conmotion/}{the same source code repository for the prototype} using the open-source \hyperlink{https://pandoc.org/}{pandoc utility}.
            \item I developed \hyperlink{https://github.com/aj-stein/conmotion/blob/develop/.github/workflows/cd.yml}{an automation pipeline to execute the publication workflow for the specification remotely in GitHub Actions} to make the authoring process more consistent over time.
            \item I created and completed troubleshooting on a deployment pipeline for \hyperlink{https://aj-stein.github.io/conmotion/}{the final release} and \hyperlink{https://conmotion.netlify.app/}{incremental draft copies} of the architecture specification. The latter is helpful to streamline feedback with multiple advisors with different copies of changes in parallel. I will also leverage this tooling and infrastructure to manage later deployments of the transparency service prototypes, greatly improving productivity. 
            \item Due to the custom needs of how I deployed pandoc, I created a \hyperlink{https://github.com/users/aj-stein/packages/container/package/pandoc\%2Flatex}{customized container image} for building the website and a PDF copy.
            \item I extended my publication workflow to generate images from architecture diagrams in \hyperlink{https://mermaid.js.org/}{the Mermaid family of declarative domain-specific languages for diagrams}.
            \item I extended my publication workflow to automatically convert the architecture specification webpage into a PDF document and cross-link to it from the webpage to better suit the preferences of several of my advisors.
        \end{itemize}
    \item I wrote several sections of \hyperlink{https://add-architecture-draft--conmotion.netlify.app/architecture.html}{the architecture specification draft}. I have not completed it on schedule due to the troubleshooting described above.
    \item For authoring the specification and preparing to develop the core of the prototype, I reviewed the sources below in more detail, modifying the outline and specification accordingly.
        \begin{itemize}
            \item transparency service specifications (e.g. \hyperlink{https://datatracker.ietf.org/doc/draft-ietf-scitt-architecture/}{SCITT}; \hyperlink{https://c2sp.org/static-ct-api}{C2SP Static Certificate Transparency API}) and industry analysis of their efficacy (for  \hyperlink{https://github.com/aj-stein/practicum_proposal/blob/main/paper.pdf}{deliverable \#3})
            \item taxonomies and models for auditing and monitoring (for  \hyperlink{https://github.com/aj-stein/practicum_proposal/blob/main/paper.pdf}{deliverable \#2})
        \end{itemize}
    \item I began the development of initial framework code and test harness for the primary component of the data service, but did not complete full development of the transparency service code.
    \item I continued work on outline for the critical analysis of FedRAMP, but it is not yet complete.
\end{enumerate}

\section*{Tasks for the Next Project Report}

In the next two weeks, I will focus on the following goals. I have sorted them in order of priority. I will have to pick up missed milestones for weeks three and four due to the troubleshooting described above.

\begin{enumerate}
    \item Ramp up development of submission API for cloud service providers and external third-party auditors.
    \item Complete the architecture specification in full.
    \item Complete outline of FedRAMP critical analysis.
    \item Start draft of FedRAMP critical analysis and potentially complete it so I can request feedback from advisors.
\end{enumerate}

\section*{Questions or issues I'm having}

\subsection*{Evaluation and Measurement}

\begin{enumerate}
    \item For more consistency, objectivity, and productivity with sentiment analysis, I am considering using hosted (e.g. OpenAI ChatGPT, Anthropic Claude) or preferably self-hosted (e.g. ollama and open-weight models) LLM tools to analyze the content in lieu of developing my own Python code. Is this permissible? If so, how do I correctly document and cite this analysis?
    \item Many security professionals do not formally criticize FedRAMP and other programs that require continuous monitoring ``on the record'' with their name and affiliation in peer-reviewed journals or in the mainstream media. However, they frequently write about their frustrations with these programs on social media, primarily by name on LinkedIn and with pseudonyms on platforms like Reddit, Bluesky, and X/Twitter. Is it permissible to consider these as primary sources for quotations? Is it permissible to use them for sentiment analysis?
    \item Some users post statistics on aggregated responses to polls on related topics without attribution or details about the respondents. Is it permissible for me to use these polls in my deliverables and final report? Am I allowed to create such polls myself and use them in the final report?
\end{enumerate}

\section*{Methodology Paragraph Summary}

For this project, I will use multiple methods to implement an alternative architecture for monitoring cloud services and modeling its potential impact. To start, I will use a quantitative and qualitative analysis of the current shortcomings and gaps for the current FedRAMP Continuous Monitoring Program. This will be the primary example of centralized continuous monitoring for which I design my mutual monitoring model for comparison. For qualitative analysis, I can perform textual analysis and sentiment analysis. I will leverage academic research, industry analysis, and a new primary source: FedRAMP's web-based forums for \hyperlink{https://www.fedramp.gov/20x/working-groups/}{the 20x reform initiative and its community working groups}. In these forums, stakeholders discuss their praise and criticism of current centralized processes and plans for future ones, often summarizing their pain points highly relevant to designing an alternative process. In addition, I will use publicly available information from FedRAMP and industry analysis to quantify the burden of the current FedRAMP Continuous Monitoring and its manual workflow. As I build a prototype based on my architecture, I will design several use cases to estimate the cost and resource efficiency to compare those costs against the estimated costs for my solution. In addition to these methods, I will use advisors familiar with FedRAMP from different stakeholder perspectives to validate information or analysis where these methods prove lacking and leave gaps.

\section*{Timeline}

\begin{xltabular}{\textwidth}{|l|X|l|}
    % \caption{Timeline for Mutual Monitoring Project} 
    % \label{tab:timeline} \\
    \hline \multicolumn{1}{|c|}{\textbf{Week \#}} & \multicolumn{1}{c|}{\textbf{Description of Task}} & \multicolumn{1}{c|}{\textbf{Status}} \\
    \endfirsthead
    \hline
    % \multicolumn{3}{c}%
    % {\tablename\ \thetable{} -- continued from previous page} \\
    % \hline \multicolumn{1}{|c|}{\textbf{Week \#}} & \multicolumn{1}{c|}{\textbf{Description of Task}} & \multicolumn{1}{c|}{\textbf{Status}} \\ \hline 
    % \endhead
    % \hline \multicolumn{3}{|r|}{{Continued on next page}} \\ \hline
    % \endfoot    
    % \hline
    % \endlastfoot
    W3 (May 26 - Jun 1) & Implement data service internals and submission API. & Deferred \\
    \hline
    W3 & First draft of data service architecture specification. & Deferred \\
    \hline
    W4 (June 2-8) & Implement data service internals and submission API. & In Progress \\
    \hline
    W3 (May 26 - Jun 1) & First draft of data service architecture specification. & In Progress \\
    \hline    
    W4 & Continue outline of FedRAMP ConMon critical analysis. & In Progress \\
    \hline    
    W4 & Finalize architecture specification with advisors' reviews. & Deferred \\
    \hline
    W5 (June 9-15) & Implement data service client to submit to submission API instances. & In Progress \\
    \hline
    W5 & Complete data service internals and submission API. & Pending \\
    \hline
    W5 & Complete outline of FedRAMP ConMon critical analysis. & In Progress \\
    \hline    
    W5 & Complete FedRAMP critical analysis document. & Pending \\
    \hline
    W5 & Finalize architecture specification with advisors' reviews. & Pending \\
    \hline    
    W6 (June 16-22) & Complete data service client to submit to submission API instances. & Pending \\
    \hline
    W6 & Implement continuous monitoring quantitative processing module for API. & Pending \\
    \hline
    W6 & Design MVP continuous monitoring use cases and quantitative measurements. & Pending \\
    \hline
    W7 (June 23-29) & Complete continuous monitoring quantitative processing module for API. & Pending \\
    \hline
    W7 & Implement MVP continuous monitoring use cases in API quantitative processing module. & Pending \\
    \hline    
    W8 (June 30 - July 6) & Start prototype deployment to cloud service tenants for testing. & Pending \\
    \hline
    W9 (July 7-13) & Complete prototype deployment to cloud service tenants for testing. & Pending \\
    \hline
\end{xltabular}

\section*{Evaluation}

[Include any evaluation plans and/or results by Progress Report 4. This may expand as you finalize the report.]

\section*{Report Outline}

[Include an outline of your final report by Progress Report 4. This may expand as you finalize the report.]

\nocite{*}
\bibliographystyle{apacite}
%  Relatives path work because you initialize top-level practicum repo
%  so the paths are consistent. Clone this repo and initialize the
%  submodules and it will work.
%  https://github.com/aj-stein/practicum.git
\bibliography{../references.bib}

\section*{\centering{Appendix}}

%  Relatives path work because you initialize top-level practicum repo
%  so the paths are consistent. Clone this repo and initialize the
%  submodules and it will work.
%  https://github.com/aj-stein/practicum.git
\includepdf[pages=-]{../prototype/build/architecture.pdf}

\label{appendix:outline}
\includepdf[pages=-]{../notes.pdf}

\end{document}
